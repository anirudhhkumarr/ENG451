\documentclass[a4paper,12pt]{report}

% Title Page
\title{Effective Communication at Job Interviews}
\author{Anirudh Kumar}

\begin{document}

\maketitle

\tableofcontents

\chapter{What is this report about?}             % chapter 1
\section{Introduction}             % section 1.1
Job interview is a meeting between an eligible candidate who is looking for a position in an organisation and a recruiter who has been 
given the resposibility to select the most appropriate person for the particular job. It is an opportunity for both the aforementioned 
parties to know each other and fulfill the purpose. \\This report focusses on the candidate side of the Job interview and how should a person
present himself/herself to the interviewer in order to increase his/her chances of getting the job offer.
As the author of a best selling book on job interviews, `Matthew DeLuca` \cite{matthewluca} says--
{\bf It is not necessarily the best candidate who gets the job offer--it is most likely the best interviewee}. It deals with two
main kind of interviews Telephonic and Personal, and discusses the verbal and non verbal aspects of both.\\
I have also included two sections in the appendix which are not directly related to ones success at Job interviews
but are more of informative nature for candidates sitting in an interview.
\newpage
\section{More on Job Interviews}       % section 1.2
\subsection{Process}                   % subsection 1.2.1
The process of hiring a candidate begins with a round for shortlisting eligible candidates based on their CV or
a test conducted by the company. GD's may be scheduled for further refining the candidates, but this practice
varies for different organisations. This is followed by a series of interviews: Technical or HR. Both of these have their own specific
methods of preparation, yet the underlying principles for effective communication at both of these interviews is the same.\\
A typical job interview can last for anything around 10 minutes to as long as a few hours. It can be conducted face to face,
 i.e. a personal interview or over a telecommunication medium like a phone or over a video conference on software such as Skype.\\
A bulk of the time in spent by the recruiter in assessing the candiate by asking questions about his work history, style of work, personality,
and any other quality relevant to the position itself. A job interview is a dynamic process for both the parties involved.
An intervewee may predict to an extent whether the interview is going successfully or not and on the other hand the interviewer
may learn something new about the interviewee which was not mentioned by him prior to the interview. So,
due to the dynamic nature of interviews the thoughts and behaviour of both sides constantly changes and
this subsequently affects their later thoughts and behaviour.\\End of the interview may feature the recruiter encouraging the candiate to pose
any questions to him. Such questions are highly encouraged as they depict an individuals interest in the company and work position offered to him.\\
The process finishes after all the eligible candidates have been interviewed and the reccruiting team has assessed them.
Although the assessing may depend on some of the previous achievements or the tests given by the candiate prior to the interview, but the interview
is still remains the major contributor to the selection of a particular individual for a position.


\chapter{Classfication of a Job interview}           % chapter 2
\section{Telephonic}     % section 2.1
\subsection{Introduction}  % subsection 2.1.1
Telephonic interviews, as is clear by the name, take place over a communication medium such as telephone.
They are incresingly becoming popular in today's times as the number of applicants is growing.
\subsubsection{Advantages}
With a very large number of applicants, usually a recruiter may want to reduce the number of applicants by interviewing all of them over telephone and eliminating
some of them beforehand. Telephonic interviews are cheaper to conduct than a traditional personal interview and far more convinient
for both parties than their counterparts and are preferred by smaller companies and start-ups. They are also very useful when the interviewer is unable to be physically
present to conduct the interview, eg. him being in other state or country.
\subsubsection{Disadvantages}
But a telephonic interview has some obvious disadvantages also. If the interview is technical there is no
means to put a check on cheating by the candidate unless a video conferencing takes place. Communication problems
may hamper the conduction of a telephonic interview. In a telephonic interview, the interviewer is unable to
judge all the qualities and shortcomings of the candiate as the non-verbal part is mostly absent.
\subsubsection{}
Despite of these shortcomings, the advantages of telephonic interviews generally advocate the need to focus
on them and prepare for them.
\section{Personal}         % section 2.2
\subsection{Introduction}  % subsection 2.2.1
This is the most common and effective type of interview in terms of selecting the best individual for the job.
The candidate faces a single or a panel of interviewers face to face. 
\subsubsection{Advantages}
The interviewers analyse every non-verbal
and verbal gesture of the candidate to decide amongst the best. This allows gathering far more information
about each other for both parties and helps them to perform better. The recruiter can elicit more in-depth response
or help the candidate by filling in information. The question of cheating is also gone as the recruiter observes every move
of the candidate.
\subsubsection{Disadvantages}
Yet it has certain disadvantages too. Most people are afraid of face to face interviews and may not perform well
during the interview, even if they are best suited for the job. It also costs the oraganisation a lot of time and
money to conduct a personal interview which is not desirable for the newer companies and small start-ups.
\subsubsection{}
Personal interview is a highly formal meeting which expects adequate amount of  maturity from both sides.
Proper dress code should be observed and the interview should take place in a quiet and isolated environment.
Following common etiquettes shows the sincerety level of the candidate. After all it is a question of how much 
is a candidate able to demonstrate to the recruiter his suitability for the job.
\subsection{Classification}
A personal interview may be futher divided into these following categories based on kinds of questions asked:
\subsubsection{Technical Interview}
Sitting in the interview for a technical position obviously requires certain level of technical knowledge. The level
varies from company to company. Some interviewers may require you to solve hard and mind teasing puzzles, some may ask 
for solving a particular problem, while some may ask you to explain your technical projects to them. The bottomline is
that you need to prepare extensively for interviews of such companies. \\But your communication skills still
play a vital role here. e.g. you may know the answer to a problem but if you have trouble explaining it to the
recruiter, that counts as a negative for you. 
\subsubsection{HR Interview}
HR interview are designed specifically for testing the soft skills of the candidates. They may also 
\chapter{Telephonic Interviews}
\section{}
\subsection{}
\chapter{Personal Interviews}
\section{}
\subsection{}
\appendix
\chapter{Technical Knowledge vs. Communication Skills}
\chapter{Interviewer Biases}
\begin{thebibliography}{9}
  % type bibliography here
  \bibitem{matthewluca}
  Matthew DeLuca,
  \emph{Best Answers to the 201 Most Frequently Asked Interview Questions}
\end{thebibliography}

\end{document}